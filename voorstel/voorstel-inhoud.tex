%---------- Inleiding ---------------------------------------------------------

% TODO: Is dit voorstel gebaseerd op een paper van Research Methods die je
% vorig jaar hebt ingediend? Heb je daarbij eventueel samengewerkt met een
% andere student?
% Zo ja, haal dan de tekst hieronder uit commentaar en pas aan.

% Dit is NIET gebaseerd op een paper van RM

%\paragraph{Opmerking}

% Dit voorstel is gebaseerd op het onderzoeksvoorstel dat werd geschreven in het
% kader van het vak Research Methods dat ik (vorig/dit) academiejaar heb
% uitgewerkt (met medesturent VOORNAAM NAAM als mede-auteur).
% 

\section{Inleiding}%
\label{sec:inleiding}

% Waarover zal je bachelorproef gaan? Introduceer het thema en zorg dat volgende zaken zeker duidelijk aanwezig zijn:

% \begin{itemize}
%   \item kaderen thema
%   \item de doelgroep
%   \item de probleemstelling en (centrale) onderzoeksvraag
%   \item de onderzoeksdoelstelling
% \end{itemize}

% Denk er aan: een typische bachelorproef is \textit{toegepast onderzoek}, wat betekent dat je start vanuit een concrete probleemsituatie in bedrijfscontext, een \textbf{casus}. Het is belangrijk om je onderwerp goed af te bakenen: je gaat voor die \textit{ene specifieke probleemsituatie} op zoek naar een goede oplossing, op basis van de huidige kennis in het vakgebied.

% De doelgroep moet ook concreet en duidelijk zijn, dus geen algemene of vaag gedefinieerde groepen zoals \emph{bedrijven}, \emph{developers}, \emph{Vlamingen}, enz. Je richt je in elk geval op it-professionals, een bachelorproef is geen populariserende tekst. Eén specifiek bedrijf (die te maken hebben met een concrete probleemsituatie) is dus beter dan \emph{bedrijven} in het algemeen.

% Formuleer duidelijk de onderzoeksvraag! De begeleiders lezen nog steeds te veel voorstellen waarin we geen onderzoeksvraag terugvinden.

% Schrijf ook iets over de doelstelling. Wat zie je als het concrete eindresultaat van je onderzoek, naast de uitgeschreven scriptie? Is het een proof-of-concept, een rapport met aanbevelingen, \ldots Met welk eindresultaat kan je je bachelorproef als een succes beschouwen?

Sportorganisaties zoals sportclubs en federaties verzamelen vandaag de dag steeds grotere hoeveelheden data. Deze gegevens omvatten wedstrijdstatistieken, medische en fysieke meetwaarden, trainingsvolumes, GPS-tracking en wedstrijdbeelden.
Uit al bestaande literatuur blijkt echter dat veel organisaties deze data niet of slechts gedeeltelijk benutten, waardoor waardevolle inzichten verloren gaan ~\autocite{Fried2016}.
Dit vormt een probleem aangezien datagedreven werken ook in de sportsector steeds belangrijker wordt voor zowel prestatieoptimalisatie als operationele besluitvorming.

De doelgroep van dit onderzoek bestaat uit sportclubs en sportfederaties die reeds data hebben verzameld of data verzamelen, maar moeilijkheden ervaren bij het verwerken, analyseren of interpreteren van deze data.
Deze organisaties beschikken vaak niet over gespecialiseerde data-expertise of geschikte tools, waardoor de drempel naar datagedreven werken nog hoog blijft.

De centrale probleemstelling luidt dat sportorganisaties onvoldoende rendement halen uit de aanwezige data, ondanks het potentieel ervan. 
Dit leidt tot suboptimale coachingbeslissingen, beperkte strategische inzichten en een lagere efficiëntie in de werking van een club.
De onderzoeksvraag die hieruit volgt is: \textbf{Hoe kunnen sportorganisaties hun bestaande data optimaler inzetten om de kwaliteit van besluitvorming en sportprestaties te verbeteren?}

Op basis van deze onderzoeksvraag worden de volgende deelvragen geformuleerd:

\begin{itemize}
    \item \textbf{Probleemdomein:}
    \begin{enumerate}
        \item Welke soorten data verzamelen sportorganisaties momenteel en hoe worden deze gegevens gebruikt?
        \item Welke belangrijkste knelpunten ervaren sportorganisaties bij het verwerken, analyseren en interpreteren van hun data?
    \end{enumerate}

    \item \textbf{Oplossingsdomein:}
    \begin{enumerate}
        \item Welke analysetechnieken en tools zijn haalbaar voor organisaties zonder diepgaande data-expertise?
        \item Welke factoren bepalen een succesvolle en duurzame implementatie van datagedreven besluitvorming binnen sportorganisaties?
    \end{enumerate}
\end{itemize}

De doelstelling van deze bachelorproef is om een \textit{praktisch toepasbaar stappenplan} en een \textit{proof-of-concept datagedreven analysemodel} te ontwikkelen waarmee sportorganisaties hun datagebruik kunnen professionaliseren.
Het eindresultaat moet concreet toepasbaar zijn binnen een breed schaal aan sportomgevingen en voldoende technische diepgang bevatten om als AI \& Data Engineering-project te gelden.

%---------- Stand van zaken ---------------------------------------------------

\section{Literatuurstudie}%
\label{sec:literatuurstudie}

% Hier beschrijf je de \emph{state-of-the-art} rondom je gekozen onderzoeksdomein, d.w.z.\ een inleidende, doorlopende tekst over het onderzoeksdomein van je bachelorproef. Je steunt daarbij heel sterk op de professionele \emph{vakliteratuur}, en niet zozeer op populariserende teksten voor een breed publiek. Wat is de huidige stand van zaken in dit domein, en wat zijn nog eventuele open vragen (die misschien de aanleiding waren tot je onderzoeksvraag!)?

% Je mag de titel van deze sectie ook aanpassen (literatuurstudie, stand van zaken, enz.). Zijn er al gelijkaardige onderzoeken gevoerd? Wat concluderen ze? Wat is het verschil met jouw onderzoek?

% Verwijs bij elke introductie van een term of bewering over het domein naar de vakliteratuur, bijvoorbeeld~\autocite{Hykes2013}! Denk zeker goed na welke werken je refereert en waarom.

% Draag zorg voor correcte literatuurverwijzingen! Een bronvermelding hoort thuis \emph{binnen} de zin waar je je op die bron baseert, dus niet er buiten! Maak meteen een verwijzing als je gebruik maakt van een bron. Doe dit dus \emph{niet} aan het einde van een lange paragraaf. Baseer nooit teveel aansluitende tekst op eenzelfde bron.

% Als je informatie over bronnen verzamelt in JabRef, zorg er dan voor dat alle nodige info aanwezig is om de bron terug te vinden (zoals uitvoerig besproken in de lessen Research Methods).

% % Voor literatuurverwijzingen zijn er twee belangrijke commando's:
% % \autocite{KEY} => (Auteur, jaartal) Gebruik dit als de naam van de auteur
% %   geen onderdeel is van de zin.
% % \textcite{KEY} => Auteur (jaartal)  Gebruik dit als de auteursnaam wel een
% %   functie heeft in de zin (bv. ``Uit onderzoek door Doll & Hill (1954) bleek
% %   ...'')

% Je mag deze sectie nog verder onderverdelen in subsecties als dit de structuur van de tekst kan verduidelijken.

De literatuurstudie rond sportdata-analyse toont aan dat datagedreven besluitvorming al jaren in opmars is, zowel bij eliteclubs als semi-professionele organisaties.
Volgens Alamar~\autocite{Alamar2013} vormen kwantitatieve analyses een steeds belangrijker fundament voor coachingbeslissinge, selectiebeleid en wedstrijdvoorbereiding.
Toch blijft de implementatie ervan ongelijk verdeeld: waar topclubs beschikken over gespecialiseerde analisten, moeten kleine organisaties het stellen zonder de nodige expertise.

Carling, Reillu en Williams~\autocite{Carling2009} tonen aan dat prestatieanalyse in veldsporten vooral waardevol wordt wanneer er verschillende datatypes worden gecombineerd, zoals fysiologische gegevens, match-notaties en psychologische factoren.
Veel sportclubs slagen hier echter niet in door hun gebrekkige data-integratie.

Daarnaast wijzen Rein en Memmert~\autocite{Rein2016} op de uitdagingen van big data binnen de sportsector. 
Hoewel er steeds meer data beschikbaar is, blijft de kwaliteit ervan variabel en worden analysetools niet altijd correct toegepast.
Best practices tonen aan dat een iteratief proces nodig is waarbij data-infrastructuur, expertiseontwikkeling en toolselectie hand in hand gaan.

Uit deze literatuur onstaat een duidelijke leegte (lacune): er bestaat voldoende theoretische kennis, maar weinig concrete richtlijnen of stappenplannen waarmee kleinere sportorganisaties hun datagebruik structureel kunnen verbeteren.
Deze bachelorproef vult deze lacune door een praktisch toepasbare methodologie te formuleren, gebaseerd op zowel literatuur als empirisch onderzoek.

%---------- Methodologie ------------------------------------------------------
\section{Methodologie}%
\label{sec:methodologie}

% Hier beschrijf je hoe je van plan bent het onderzoek te voeren. Welke onderzoekstechniek ga je toepassen om elk van je onderzoeksvragen te beantwoorden? Gebruik je hiervoor literatuurstudie, interviews met belanghebbenden (bv.~voor requirements-analyse), experimenten, simulaties, vergelijkende studie, risico-analyse, PoC, \ldots?

% Valt je onderwerp onder één van de typische soorten bachelorproeven die besproken zijn in de lessen Research Methods (bv.\ vergelijkende studie of risico-analyse)? Zorg er dan ook voor dat we duidelijk de verschillende stappen terug vinden die we verwachten in dit soort onderzoek!

% Vermijd onderzoekstechnieken die geen objectieve, meetbare resultaten kunnen opleveren. Enquêtes, bijvoorbeeld, zijn voor een bachelorproef informatica meestal \textbf{niet geschikt}. De antwoorden zijn eerder meningen dan feiten en in de praktijk blijkt het ook bijzonder moeilijk om voldoende respondenten te vinden. Studenten die een enquête willen voeren, hebben meestal ook geen goede definitie van de populatie, waardoor ook niet kan aangetoond worden dat eventuele resultaten representatief zijn.

% Uit dit onderdeel moet duidelijk naar voor komen dat je bachelorproef ook technisch voldoen\-de diepgang zal bevatten. Het zou niet kloppen als een bachelorproef informatica ook door bv.\ een student marketing zou kunnen uitgevoerd worden.

% Je beschrijft ook al welke tools (hardware, software, diensten, \ldots) je denkt hiervoor te gebruiken of te ontwikkelen.

% Probeer ook een tijdschatting te maken. Hoe lang zal je met elke fase van je onderzoek bezig zijn en wat zijn de concrete \emph{deliverables} in elke fase?

Het onderzoek bestaat uit vier fases, die telkens gekoppeld zijn aan specifieke onderzoekstechnieken:

\subsection*{Fase 1: Analyse van de huidige situatie}
\begin{itemize}
  \item Literatuurstudie rond sportdata, datagedreven besluitvorming en bestaande frameworks.
  \item Interviews van vertegenwoordigers of belanghebbenden van minstens twee sportclubs of federaties.
  \item Analyse van de soorten data die sportorganisaties momenteel verzamelen.
\end{itemize}

\subsection*{Fase 2: Identificatie van knelpunten en behoeften}
\begin{itemize}
  \item Thematische analyse van de interviewresultaten.
  \item Evaluatie van dataflows, databronnen en aanwezige expertise.
  \item Opstellen van requirements voor een haalbaar datamodel.
\end{itemize}

\subsection*{Fase 3: Proof-of-concept en experimentele analyse}
\begin{itemize}
  \item Selectie en preprocessing van sportdata.
  \item Toepassen van datagedreven analysetools zoals:
      \begin{itemize}
          \item regressiemodellen,
          \item clustering (bv. k-means voor spelersprofielen),
          \item anomaly detection voor blessurerisico,
          \item dashboards in Python (pandas, scikit-learn, eventueel PowerBI).
          \item ...
      \end{itemize}
  \item Evaluatie van de bruikbaarheid en uitvoerbaarheid voor niet-experts.
\end{itemize}

\subsection*{Fase 4: Uitwerking van een relevant stappenplan}
\begin{itemize}
  \item Samenvatten van succesfactoren uit de literatuur.
  \item Integreren van empirische bevindingen
  \item Formuleren van een concreet stappenplan inclusief:
      \begin{itemize}
          \item databeheer,
          \item toolselectie,
          \item noodzakelijke rollen en competenties,
          \item implementatievolgorde.
      \end{itemize}
\end{itemize}

\subsection*{Tools en tijdsplanning}
De technische uitvoering gebeurt voornamelijk in Python met libraries zoals pandas, NumPy, scikit-learn en Matplotlib.
Voor datavisualisatie kan zowel PowerBI als Streamlit worden gebruikt.
De totale verwachte doorlooptijd bedraagt ongeveer 14 weken, deze is opgesplitst in onderzoek, ontwikkeling, analyse en rapportering.

%---------- Verwachte resultaten ----------------------------------------------
\section{Verwacht resultaat, conclusie}%
\label{sec:verwachte_resultaten}

% Hier beschrijf je welke resultaten je verwacht. Als je metingen en simulaties uitvoert, kan je hier al mock-ups maken van de grafieken samen met de verwachte conclusies. Benoem zeker al je assen en de onderdelen van de grafiek die je gaat gebruiken. Dit zorgt ervoor dat je concreet weet welk soort data je moet verzamelen en hoe je die moet meten.

% Wat heeft de doelgroep van je onderzoek aan het resultaat? Op welke manier zorgt jouw bachelorproef voor een meerwaarde?

% Hier beschrijf je wat je verwacht uit je onderzoek, met de motivatie waarom. Het is \textbf{niet} erg indien uit je onderzoek andere resultaten en conclusies vloeien dan dat je hier beschrijft: het is dan juist interessant om te onderzoeken waarom jouw hypothesen niet overeenkomen met de resultaten.

Het onderzoek verwacht ons een duidelijk inzicht te geven in hoe sportorganisaties hun bestaande data waardevoller kunnen inzetten, zowel op sportief als strategisch niveau.
Verwachte resultaten zijn:

\begin{itemize}
  \item Een overzicht van welke datatypes het meest impactvol zijn.
  \item Een analyse van de grootste knelpunten in huidige datapraktijken.
  \item Een proof-of-concept data eenvoudige maar effectieve data-analyses demonstreert. (Liefst gebaseerd op non-fictieve data)
  \item Een concreet stappenplan waarmee sportorganisaties hun datagebruik kunnen professionaliseren.
\end{itemize}

De meerwaarde voor de doelgroep is aanzienlijk: organisaties krijgen hanteerbare richtlijnen die hen helpen op systematisch, haalbaar en duurzaam datagedreven te werken.
Dit onderzoek draagt zo bij aan een professionelere, efficiëntere en meer onderbouwde sportwerking.
