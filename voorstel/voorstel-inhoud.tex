%---------- Inleiding ---------------------------------------------------------

% TODO: Is dit voorstel gebaseerd op een paper van Research Methods die je
% vorig jaar hebt ingediend? Heb je daarbij eventueel samengewerkt met een
% andere student?
% Zo ja, haal dan de tekst hieronder uit commentaar en pas aan.

% Dit is NIET gebaseerd op een paper van RM

%\paragraph{Opmerking}

% Dit voorstel is gebaseerd op het onderzoeksvoorstel dat werd geschreven in het
% kader van het vak Research Methods dat ik (vorig/dit) academiejaar heb
% uitgewerkt (met medesturent VOORNAAM NAAM als mede-auteur).
% 

\section{Inleiding}%
\label{sec:inleiding}

% Waarover zal je bachelorproef gaan? Introduceer het thema en zorg dat volgende zaken zeker duidelijk aanwezig zijn:

% \begin{itemize}
%   \item kaderen thema
%   \item de doelgroep
%   \item de probleemstelling en (centrale) onderzoeksvraag
%   \item de onderzoeksdoelstelling
% \end{itemize}

% Denk er aan: een typische bachelorproef is \textit{toegepast onderzoek}, wat betekent dat je start vanuit een concrete probleemsituatie in bedrijfscontext, een \textbf{casus}. Het is belangrijk om je onderwerp goed af te bakenen: je gaat voor die \textit{ene specifieke probleemsituatie} op zoek naar een goede oplossing, op basis van de huidige kennis in het vakgebied.

% De doelgroep moet ook concreet en duidelijk zijn, dus geen algemene of vaag gedefinieerde groepen zoals \emph{bedrijven}, \emph{developers}, \emph{Vlamingen}, enz. Je richt je in elk geval op it-professionals, een bachelorproef is geen populariserende tekst. Eén specifiek bedrijf (die te maken hebben met een concrete probleemsituatie) is dus beter dan \emph{bedrijven} in het algemeen.

% Formuleer duidelijk de onderzoeksvraag! De begeleiders lezen nog steeds te veel voorstellen waarin we geen onderzoeksvraag terugvinden.

% Schrijf ook iets over de doelstelling. Wat zie je als het concrete eindresultaat van je onderzoek, naast de uitgeschreven scriptie? Is het een proof-of-concept, een rapport met aanbevelingen, \ldots Met welk eindresultaat kan je je bachelorproef als een succes beschouwen?

% Sportorganisaties zoals sportclubs en federaties verzamelen vandaag de dag steeds grotere hoeveelheden data. Deze gegevens omvatten wedstrijdstatistieken, medische en fysieke meetwaarden, trainingsvolumes, GPS-tracking en wedstrijdbeelden.
% Uit al bestaande literatuur blijkt echter dat veel organisaties deze data niet of slechts gedeeltelijk benutten, waardoor waardevolle inzichten verloren gaan ~\autocite{Fried2016}.
% Dit vormt een probleem aangezien datagedreven werken ook in de sportsector steeds belangrijker wordt voor zowel prestatieoptimalisatie als operationele besluitvorming.

% De doelgroep van dit onderzoek bestaat uit sportclubs en sportfederaties die reeds data hebben verzameld of data verzamelen, maar moeilijkheden ervaren bij het verwerken, analyseren of interpreteren van deze data.
% Deze organisaties beschikken vaak niet over gespecialiseerde data-expertise of geschikte tools, waardoor de drempel naar datagedreven werken nog hoog blijft.

% \textcolor{probleemkleur}{De centrale probleemstelling luidt dat sportorganisaties onvoldoende rendement halen uit de aanwezige data, ondanks het potentieel ervan.}
% Dit leidt tot suboptimale coachingbeslissingen, beperkte strategische inzichten en een lagere efficiëntie in de werking van een club.
% \textcolor{hoofdvraagkleur}{De onderzoeksvraag die hieruit volgt is: \textbf{Hoe kunnen sportorganisaties hun bestaande data optimaler inzetten om de kwaliteit van besluitvorming en sportprestaties te verbeteren?}}

Voetbalclubs verzamelen vandaag de dag steeds grotere hoeveelheden data. Deze gegevens omvatten wedstrijdstatistieken, medische en fysieke meetwaarden, trainingsvolumes, GPS-tracking en wedstrijdbeelden.
Uit al bestaande literatuur blijkt echter dat veel voetbalclubs deze data niet of slechts gedeeltelijk benutten, waardoor waardevolle inzichten verloren gaan ~\autocite{Fried2016}.
Dit vormt een probleem aangezien datagedreven werken ook binnen het voetbal steeds belangrijker wordt voor zowel prestatieoptimalisatie als operationele besluitvorming.

De doelgroep van dit onderzoek bestaat uit voetbalclubs die reeds data hebben verzameld of data verzamelen, maar moeilijkheden ervaren bij het verwerken, analyseren of interpreteren van deze data.
Deze clubs beschikken vaak niet over gespecialiseerde data-expertise of geschikte tools, waardoor de drempel naar datagedreven werken nog hoog blijft.

\textcolor{probleemkleur}{De centrale probleemstelling luidt dat voetbalclubs onvoldoende rendement halen uit de aanwezige data, ondanks het potentieel ervan.}
Dit leidt tot suboptimale coachingbeslissingen, beperkte strategische inzichten en een lagere efficiëntie in de werking van een club.
\textcolor{hoofdvraagkleur}{De onderzoeksvraag die hieruit volgt is: \textbf{Hoe kunnen voetbalclubs hun bestaande data optimaler inzetten om de kwaliteit van besluitvorming en sportprestaties te verbeteren?}}

% Op basis van deze onderzoeksvraag worden de volgende deelvragen geformuleerd:

% \begin{itemize}
%     \item \textbf{Probleemdomein:}
%     \begin{enumerate}
%         \item Welke soorten data verzamelen sportorganisaties momenteel en hoe worden deze gegevens gebruikt?
%         \item Welke belangrijkste knelpunten ervaren sportorganisaties bij het verwerken, analyseren en interpreteren van hun data?
%     \end{enumerate}

%     \item \textbf{Oplossingsdomein:}
%     \begin{enumerate}
%         \item Welke analysetechnieken en tools zijn haalbaar voor organisaties zonder diepgaande data-expertise?
%         \item Welke factoren bepalen een succesvolle en duurzame implementatie van datagedreven besluitvorming binnen sportorganisaties?
%     \end{enumerate}
% \end{itemize}

% Op basis van deze onderzoeksvraag worden verschillende deelvragen geformuleerd die het onderzoek wat meer richting geven.
% \textcolor{deelvraagkleur}{In het probleemdomein wordt eerst nagegaan welke soorten data sportorganisaties vandaag verzamelen en op welke manier deze gegevens in de praktijk worden ingezet.}
% Daarbij wordt onderzocht in welke mate organisaties gebruikmaken van wedstrijdstatistieken, fysieke en medische metingen, GPS-tracking of andere vormen van prestatie- en contextdata.
% \textcolor{deelvraagkleur}{Vervolgens wordt aandacht besteed aan de voornaamste knelpunten die sportorganisaties ervaren bij het verwerken, analyseren en interpreteren van hun data.}
% Dit omvat zowel technische beperkingen, zoals datakwaliteit of fragmentatie, als organisatorische uitdagingen zoals het ontbreken van expertise of geschikte tools.

% \textcolor{deelvraagkleur}{Binnen het oplossingsdomein richt het onderzoek zich op de vraag welke analysetechnieken en tools haalbaar zijn voor organisaties die niet beschikken over diepgaande data-ervaring.}
% Hierbij wordt bekeken welke methoden voldoende eenvoudig, robuust en praktisch bruikbaar zijn om in een realistische sportcontext toegepast te worden.
% \textcolor{deelvraagkleur}{Tot slot wordt onderzocht welke factoren bepalend zijn voor een succesvolle en duurzame implementatie van datagedreven besluitvorming binnen sportorganisaties.}
% Hierbij gaat het zowel om organisatorische voorwaarden, zoals rollen en processen, als om technische aspecten, waaronder databeheer en toolselectie.

% De doelstelling van deze bachelorproef is om een \textit{praktisch toepasbaar stappenplan} en een \textit{proof-of-concept datagedreven analysemodel} te ontwikkelen waarmee sportorganisaties hun datagebruik kunnen professionaliseren.
% Het eindresultaat moet concreet toepasbaar zijn binnen een breed schaal aan sportomgevingen en voldoende technische diepgang bevatten om als AI \& Data Engineering-project te gelden.

Op basis van deze onderzoeksvraag worden verschillende deelvragen geformuleerd die het onderzoek wat meer richting geven.
\textcolor{deelvraagkleur}{In het probleemdomein wordt eerst nagegaan welke soorten data voetbalclubs vandaag verzamelen en op welke manier deze gegevens in de praktijk worden ingezet.}
Daarbij wordt onderzocht in welke mate clubs gebruikmaken van wedstrijdstatistieken, fysieke en medische metingen, GPS-tracking of andere vormen van prestatie- en contextdata.
\textcolor{deelvraagkleur}{Vervolgens wordt aandacht besteed aan de voornaamste knelpunten die voetbalclubs ervaren bij het verwerken, analyseren en interpreteren van hun data.}
Dit omvat zowel technische beperkingen, zoals datakwaliteit of fragmentatie, als organisatorische uitdagingen zoals het ontbreken van expertise of geschikte tools.

\textcolor{deelvraagkleur}{Binnen het oplossingsdomein richt het onderzoek zich op de vraag welke analysetechnieken en tools haalbaar zijn voor voetbalclubs die niet beschikken over diepgaande data-ervaring.}
Hierbij wordt bekeken welke methoden voldoende eenvoudig, robuust en praktisch bruikbaar zijn om in een realistische voetbalcontext toegepast te worden.
\textcolor{deelvraagkleur}{Tot slot wordt onderzocht welke factoren bepalend zijn voor een succesvolle en duurzame implementatie van datagedreven besluitvorming binnen voetbalclubs.}
Hierbij gaat het zowel om organisatorische voorwaarden, zoals rollen en processen, als om technische aspecten, waaronder databeheer en toolselectie.

De doelstelling van deze bachelorproef is om een \textit{praktisch toepasbaar stappenplan} en een \textit{proof-of-concept datagedreven analysemodel} te ontwikkelen waarmee voetbalclubs hun datagebruik kunnen professionaliseren.
Het eindresultaat moet concreet toepasbaar zijn binnen een breed schaal aan voetbalomgevingen en voldoende technische diepgang bevatten om als AI \& Data Engineering-project te gelden.

%---------- Stand van zaken ---------------------------------------------------

\section{Literatuurstudie}%
\label{sec:literatuurstudie}

% Hier beschrijf je de \emph{state-of-the-art} rondom je gekozen onderzoeksdomein, d.w.z.\ een inleidende, doorlopende tekst over het onderzoeksdomein van je bachelorproef. Je steunt daarbij heel sterk op de professionele \emph{vakliteratuur}, en niet zozeer op populariserende teksten voor een breed publiek. Wat is de huidige stand van zaken in dit domein, en wat zijn nog eventuele open vragen (die misschien de aanleiding waren tot je onderzoeksvraag!)?

% Je mag de titel van deze sectie ook aanpassen (literatuurstudie, stand van zaken, enz.). Zijn er al gelijkaardige onderzoeken gevoerd? Wat concluderen ze? Wat is het verschil met jouw onderzoek?

% Verwijs bij elke introductie van een term of bewering over het domein naar de vakliteratuur, bijvoorbeeld~\autocite{Hykes2013}! Denk zeker goed na welke werken je refereert en waarom.

% Draag zorg voor correcte literatuurverwijzingen! Een bronvermelding hoort thuis \emph{binnen} de zin waar je je op die bron baseert, dus niet er buiten! Maak meteen een verwijzing als je gebruik maakt van een bron. Doe dit dus \emph{niet} aan het einde van een lange paragraaf. Baseer nooit teveel aansluitende tekst op eenzelfde bron.

% Als je informatie over bronnen verzamelt in JabRef, zorg er dan voor dat alle nodige info aanwezig is om de bron terug te vinden (zoals uitvoerig besproken in de lessen Research Methods).

% % Voor literatuurverwijzingen zijn er twee belangrijke commando's:
% % \autocite{KEY} => (Auteur, jaartal) Gebruik dit als de naam van de auteur
% %   geen onderdeel is van de zin.
% % \textcite{KEY} => Auteur (jaartal)  Gebruik dit als de auteursnaam wel een
% %   functie heeft in de zin (bv. ``Uit onderzoek door Doll & Hill (1954) bleek
% %   ...'')

% Je mag deze sectie nog verder onderverdelen in subsecties als dit de structuur van de tekst kan verduidelijken.

De literatuurstudie rond sportdata-analyse toont aan dat datagedreven besluitvorming al jaren in opmars is, zowel bij eliteclubs als semi-professionele organisaties.
Volgens Alamar~\autocite{Alamar2013} vormen kwantitatieve analyses een steeds belangrijker fundament voor coachingbeslissingen, selectiebeleid en wedstrijdvoorbereiding.
Toch blijft de implementatie ervan ongelijk verdeeld: waar topclubs beschikken over gespecialiseerde analisten, moeten kleine organisaties het stellen zonder de nodige expertise.

Carling, Reilly en Williams~\autocite{Carling2009} tonen aan dat prestatieanalyse in veldsporten vooral waardevol wordt wanneer er verschillende datatypes worden gecombineerd, zoals fysiologische gegevens, match-notaties en psychologische factoren.
Veel sportclubs slagen hier echter niet in door hun gebrekkige data-integratie.

Daarnaast wijzen Rein en Memmert~\autocite{Rein2016} op de uitdagingen van big data binnen de sportsector. 
Hoewel er steeds meer data beschikbaar is, blijft de kwaliteit ervan variabel en worden analysetools niet altijd correct toegepast.
Best practices tonen aan dat een iteratief proces nodig is waarbij data-infrastructuur, expertiseontwikkeling en toolselectie hand in hand gaan.

Uit deze literatuur ontstaat een duidelijke leegte (lacune): er bestaat voldoende theoretische kennis, maar weinig concrete richtlijnen of stappenplannen waarmee kleinere sportorganisaties hun datagebruik structureel kunnen verbeteren.
Deze bachelorproef vult deze lacune door een praktisch toepasbare methodologie te formuleren, gebaseerd op zowel literatuur als empirisch onderzoek.

De opkomst van sports analytics is de voorbije tien tot vijftien jaar versneld door betere data-verzamelingsmethoden (GPS, wearables, tracking), grotere rekenkracht en toegankelijkere analysetools.
Vroege overzichten benadrukken dat kwantitatieve analyses belangrijke ondersteunende informatie leveren voor coaching, selectie en tactische beslissingen, en dat deze analyses zowel sportieve als organisatorische meerwaarde kunnen bieden \autocite{Alamar2013,Fried2016}. 

Een eerste belangrijke rode draad in de literatuur behandelt de \textbf{typen data} die beschikbaar zijn binnen sportorganisaties en de meerwaarde daarvan.
Traditionele bronnen zijn wedstrijdstatistieken en notaties. 
Recentere bronnen omvatten positie- en trackingdata, fysiologische metingen uit wearables en medische data. 
Carling, Reilly en Williams tonen aan dat de meerwaarde vooral ontstaat wanneer meerdere datatypes gecombineerd worden (match-notatie, fysiologie, psychologische indicatoren) en correct geïnterpreteerd \autocite{Carling2009}.
Rein en Memmert benadrukken dat positionele (tracking) data en 'big data' methoden de mogelijkheden voor diepere tactische analyses sterk uitbreiden, maar dat dit ook nieuwe challenges met zich meebrengt op vlak van data-kwaliteit en verwerking \autocite{Rein2016}.

Een tweede rode draad in de literatuur onderzoekt de \textbf{implementatieproblemen in kleinere organisaties}.
Recent onderzoek en masterscripties laten zien dat veel kleine en semi-professionele clubs structurele beperkingen hebben: beperkte financiële middelen, gebrek aan technische expertise, en vaak gefragmenteerde datastromen.
Studies die case-based onderzoeken uitvoeren in kleinere clubs concluderen dat succesvolle toepassing van analytics vaak afhangt van het opbouwen van procesmatige vaardigheden en routines om data in te zetten en van het ontwerpen van lichtgewicht, herhaalbare analysemethoden die aansluiten bij de operationele realiteit van de club \autocite{Scoring2024,Postma2023}.
Deze studies tonen aan dat top-down investering in infrastructuur alleen zelden volstaat; in plaats daarvan is een pragmatische, gefaseerde aanpak nodig waarbij laagdrempelige tools en duidelijke use-cases (bijv. blessurerisico of spelersevaluatie) centraal staan.

Tenslotte is er literatuur die expliciet ingaat op \textbf{praktische stappen en best practices} voor het opzetten van een proof-of-concept.
Kaderwerken en studiewerkstukken raden aan te starten met een heldere probleemdefinitie, het identificeren van beschikbare datasets, een simpele baseline-analyse (descriptive statistics, eenvoudige regressies/clustering) en pas daarna iteratief complexere modellen te introduceren.
Dit iteratieve, 'start small, scale up'-principe is terug te vinden in meerdere case studies en vormt een brug tussen academische methodes en toepasbare adviezen voor clubs zonder uitgebreide resources \autocite{Postma2023,Scoring2024}.

%---------- Methodologie ------------------------------------------------------
\section{Methodologie}%
\label{sec:methodologie}

% Hier beschrijf je hoe je van plan bent het onderzoek te voeren. Welke onderzoekstechniek ga je toepassen om elk van je onderzoeksvragen te beantwoorden? Gebruik je hiervoor literatuurstudie, interviews met belanghebbenden (bv.~voor requirements-analyse), experimenten, simulaties, vergelijkende studie, risico-analyse, PoC, \ldots?

% Valt je onderwerp onder één van de typische soorten bachelorproeven die besproken zijn in de lessen Research Methods (bv.\ vergelijkende studie of risico-analyse)? Zorg er dan ook voor dat we duidelijk de verschillende stappen terug vinden die we verwachten in dit soort onderzoek!

% Vermijd onderzoekstechnieken die geen objectieve, meetbare resultaten kunnen opleveren. Enquêtes, bijvoorbeeld, zijn voor een bachelorproef informatica meestal \textbf{niet geschikt}. De antwoorden zijn eerder meningen dan feiten en in de praktijk blijkt het ook bijzonder moeilijk om voldoende respondenten te vinden. Studenten die een enquête willen voeren, hebben meestal ook geen goede definitie van de populatie, waardoor ook niet kan aangetoond worden dat eventuele resultaten representatief zijn.

% Uit dit onderdeel moet duidelijk naar voor komen dat je bachelorproef ook technisch voldoen\-de diepgang zal bevatten. Het zou niet kloppen als een bachelorproef informatica ook door bv.\ een student marketing zou kunnen uitgevoerd worden.

% Je beschrijft ook al welke tools (hardware, software, diensten, \ldots) je denkt hiervoor te gebruiken of te ontwikkelen.

% Probeer ook een tijdschatting te maken. Hoe lang zal je met elke fase van je onderzoek bezig zijn en wat zijn de concrete \emph{deliverables} in elke fase?

Het onderzoek bestaat uit vier fases, die telkens gekoppeld zijn aan specifieke onderzoekstechnieken:

\subsection*{Fase 1: Analyse van de huidige situatie}
% \begin{itemize}
%   \item Literatuurstudie rond sportdata, datagedreven besluitvorming en bestaande frameworks.
%   \item Interviews van vertegenwoordigers of belanghebbenden van minstens twee sportclubs of federaties.
%   \item Analyse van de soorten data die sportorganisaties momenteel verzamelen.
% \end{itemize}
In de eerste fase wordt de huidige situatie binnen sportorganisaties onderzocht.
Dit gebeurt door een uitgebreide literatuurstudie uit te voeren naar bestaande inzichten rond sportdata, datagedreven besluitvorming en relevante analytische frameworks.
Deze theoretische basis wordt aangevuld met praktijkinformatie door gesprekken te voeren met vertegenwoordigers of belanghebbenden van minstens twee sportclubs of sportfederaties.
Tijdens deze interviews wordt nagegaan welke soorten data momenteel worden verzameld, voor welke doeleinden deze gegevens worden gebruikt en in welke mate organisaties reeds ervaring hebben met datagebaseerde besluitvorming.
Door deze combinatie van literatuur en praktijkobservaties ontstaat een helder beeld van de uitgangspositie en van de mate waarop data vandaag structureel worden ingezet.

\subsection*{Fase 2: Identificatie van knelpunten en behoeften}
% \begin{itemize}
%   \item Thematische analyse van de interviewresultaten.
%   \item Evaluatie van dataflows, databronnen en aanwezige expertise.
%   \item Opstellen van requirements voor een haalbaar datamodel.
% \end{itemize}
In de tweede fase ligt de nadruk op het identificeren van de voornaamste knelpunten en behoeften die in de interviews naar voren komen.
De verzamelde informatie wordt thematisch geanalyseerd, waarna een evaluatie wordt gemaakt van bestaande dataflows, databronnen en de aanwezige expertise binnen de betrokken organisaties.
Op basis van deze analyse worden zowel de functionele als technische requirements opgesteld waaraan een haalbaar datamodel of analyseproces moet voldoen.
Deze fase vormt een cruciale overgang tussen het begrijpen van de huidige situatie en de ontwikkeling van een concreet, technisch onderbouwde oplossing.

\subsection*{Fase 3: Proof-of-concept en experimentele analyse}
% \begin{itemize}
%   \item Selectie en preprocessing van sportdata.
%   \item Toepassen van datagedreven analysetools zoals:
%       \begin{itemize}
%           \item regressiemodellen,
%           \item clustering (bv. k-means voor spelersprofielen),
%           \item anomaly detection voor blessurerisico,
%           \item dashboards in Python (pandas, scikit-learn, eventueel PowerBI).
%           \item ...
%       \end{itemize}
%   \item Evaluatie van de bruikbaarheid en uitvoerbaarheid voor niet-experts.
% \end{itemize}
De derde fase bestaat uit de ontwikkeling van een proof-of-concept waarin wordt onderzocht hoe sportdata op een praktische en haalbare manier kan worden geanalyseerd.
Eerst wordt relevante data geselecteerd en voorbewerkt, zodat deze klaar is voor analyse.
Vervolgens worden verschillende analysetechnieken toegepast, waaronder regressiemodellen, clusteringmethoden zoals k-means voor het onderscheiden van spelersprofielen, en anomaly detection om mogelijke blessurerisico's te identificeren.
De resultaten worden ondersteund door visualisaties en dashboards die worden ontwikkeld in Python met behulp van onder meer pandas, NumPy, scikit-learn en Matplotlib.
Waar nodig wordt aanvullende tooling zoals PowerBI of Streamlit gebruikt om de interpretatie te verbeteren.
Deze fase heeft als doel om inzicht te geven in welke technieken haalbaar, begrijpelijk en bruikbaar zijn voor organisaties die niet over diepgaande data-expertise beschikken.

\subsection*{Fase 4: Uitwerking van een relevant stappenplan}
% \begin{itemize}
%   \item Samenvatten van succesfactoren uit de literatuur.
%   \item Integreren van empirische bevindingen
%   \item Formuleren van een concreet stappenplan inclusief:
%       \begin{itemize}
%           \item databeheer,
%           \item toolselectie,
%           \item noodzakelijke rollen en competenties,
%           \item implementatievolgorde.
%       \end{itemize}
% \end{itemize}
In de vierde en laatste fase wordt op basis van zowel de literatuur als de empirische bevindingen een concreet en toepasbaar stappenplan opgesteld dat sportorganisaties moet helpen hun datagebruik te professionaliseren.
Dit stappenplan bevat richtlijnen voor databeheer, de selectie van geschikte tools, de rollen en competenties die vereist zijn binnen de organisatie en een realistische volgorde waarin datagedreven processen kunnen worden geïmplementeerd.
Door theorie en praktijk te integreren, resulteert deze fase in een methode die zowel technisch onderbouwd als praktisch haalbaar is voor kleine en middelgrote sportomgevingen.

\subsection*{Tools en tijdsplanning}
De technische uitvoering gebeurt voornamelijk in Python met libraries zoals pandas, NumPy, scikit-learn en Matplotlib.
Voor datavisualisatie kan zowel PowerBI als Streamlit worden gebruikt.
De totale verwachte doorlooptijd bedraagt ongeveer veertien weken, deze is opgesplitst in onderzoek, ontwikkeling, analyse en rapportering.

%---------- Verwachte resultaten ----------------------------------------------
\section{Verwacht resultaat, conclusie}%
\label{sec:verwachte_resultaten}

% Hier beschrijf je welke resultaten je verwacht. Als je metingen en simulaties uitvoert, kan je hier al mock-ups maken van de grafieken samen met de verwachte conclusies. Benoem zeker al je assen en de onderdelen van de grafiek die je gaat gebruiken. Dit zorgt ervoor dat je concreet weet welk soort data je moet verzamelen en hoe je die moet meten.

% Wat heeft de doelgroep van je onderzoek aan het resultaat? Op welke manier zorgt jouw bachelorproef voor een meerwaarde?

% Hier beschrijf je wat je verwacht uit je onderzoek, met de motivatie waarom. Het is \textbf{niet} erg indien uit je onderzoek andere resultaten en conclusies vloeien dan dat je hier beschrijft: het is dan juist interessant om te onderzoeken waarom jouw hypothesen niet overeenkomen met de resultaten.


% Verwachte resultaten zijn:

% \begin{itemize}
%   \item Een overzicht van welke datatypes het meest impactvol zijn.
%   \item Een analyse van de grootste knelpunten in huidige datapraktijken.
%   \item Een proof-of-concept data eenvoudige maar effectieve data-analyses demonstreert. (Liefst gebaseerd op non-fictieve data)
%   \item Een concreet stappenplan waarmee sportorganisaties hun datagebruik kunnen professionaliseren.
% \end{itemize}


Het onderzoek wordt verwacht een helder en diepgaand inzicht op te leveren in de manier waarop sportorganisaties hun bestaande data efficiënter en effectiever kunnen benutten, zowel voor sportieve doeleinden als voor strategische besluitvorming.
Tijdens de literatuurstudie en de analyse van de huidige praktijken wordt duidelijk welke soorten data vandaag de grootste impact hebben op prestaties, evaluaties en operationele keuzes binnen sportomgevingen.
Het resultaat van deze fase bestaat uit een overzicht van de meest relevante datatypes en een analyse van de manier waarop deze in de praktijk worden gebruikt of onderbenut.

Daarnaast zal het onderzoek een gedetailleerd beeld schetsen van de voornaamste knelpunten waarmee sportorganisaties te maken hebben bij het verwerken, analyseren en interpreteren van hun data.
Dit omvat zowel technische uitdagingen, zoals de kwaliteit en structuur van beschikbare datasets, als organisatorische factoren, waaronder beperkte expertise, onvoldoende tooling of gefragmenteerde dataflows.
Door deze uitdagingen in kaart te brengen ontstaat een duidelijk zicht op de voorwaarden die nodig zijn om datagedreven werken succesvol te implementeren.

De ontwikkeling van het proof-of-concept vormt een cruciaal onderdeel van de verwachte resultaten.
Dit prototype zal aantonen hoe eenvoudige maar goed gekozen analysetechnieken kunnen worden ingezet om concrete en bruikbare inzichten te genereren op basis van sportdata.
De bedoeling is dat dit proof-of-concept gebaseerd is op realistische of volledig non-fictieve datasets, zodat het model daadwerkelijk kan tonen hoe data binnen een organisatie gestructureerd, geanalyseerd en geïnterpreteerd kan worden.
Hierdoor wordt de praktische haalbaarheid van datagedreven werken tastbaar gemaakt, ook voor organisaties die geen uitgebreide technische infrastructuur of expertise bezitten.

Tot slot resulteert het onderzoek in een concreet en toepasbaar stappenplan dat sportorganisaties ondersteunt bij het professionaliseren van hun datagebruik.
Dit stappenplan biedt richtlijnen voor databeheer, toolselectie, procesinrichting en de ontwikkeling van benodigde competenties.
Door de combinatie van theoretische inzichten, empirische bevindingen en technische validatie biedt het onderzoek een waardevolle leidraad voor sportorganisaties die op een systematische, haalbare en duurzame manier datagedreven willen gaan werken.

De meerwaarde voor de doelgroep is dan ook aanzienlijk.
Sportclubs en sportfederaties krijgen toegang tot een duidelijk, onderbouwd en praktisch toepasbaar kader dat hen in staat stelt betere beslissingen te nemen, hun werking efficiënter te organiseren en hun sportieve prestaties te optimaliseren.
Op die manier draagt het onderzoek bij aan een professionelere en meer datagedreven sportsector.
